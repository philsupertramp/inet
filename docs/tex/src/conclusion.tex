\section{Conclusion}\label{sec:conclusion}
To conclude, the here introduced CNN architecture has its limitations and existing state-of-the-art classifiers like MobileNet perform predictions with higher accuracy.
But YOLOv5 seems to be a good candidate for further investigations, experiments and potentially to be the model that is later applied in \textit{KInsecta}'s monitoring device.

First, it should be said that it was extremely fascinating and enjoyable to research this paper.
To put it into the words of my generation: "I had the blast of my life!".\\
But the overall task given for this paper has been daunting for two reasons.
Mainly, the three month time constraint limiting this thesis was a very tight schedule, considering that knowledge about most techniques used here was required to be acquired.
Secondly, the object detection task itself was very challenging.
On one hand, due to the complexity of CNNs and the abstraction through code, using Tensorflow.
On the other, because it took a long time to achieve the here described results\footnote{Some of the final results were computed in the last week prior to submission.}.
The results in \nameref{sec:application} are a great building block for my processor at \textit{KInsecta}, and give me enough confidence to claim that the task will eventually be solved in a satisfying manner.

\subsection{General ideas and prospects for similar problems}\label{subsec:prospects}
For ML projects in general, it is advised to get a good understanding of the problem and the available data for it.
It seemed to be one of the most crucial parts of the ML project pipeline, introduced in the beginning of this paper (\figref{fig:pipeline}).
When enough knowledge about the raw input data is collected, augmentation techniques should be assembled to increase the amount of samples in the training set.
After understanding and increasing the data, it is also recommend to use SotA (state-of-the-art) architectures first.
These architectures exist for a reason, they generalize well on many different data sets.
The chance of picking a problem that can not be solved in a satisfying way, using these models, seems from the here obtained results, unlikely.
Additionally, great predictions with SotA architectures are a good motivation and can reduce the time spent exploring into a specific direction, rather than trying out a different ones.\\
If the objective is still to create a custom architecture from ground up, the reader is recommended to research different techniques and architectures thoroughly prior to performing first implementation attempts.
This can also resolve frustration and might lead into finding comparable results from other researchers.\\
By now, the reader should be motivated to look deeper into YOLO, especially YOLOv5.
\subsection{Outlook}\label{subsec:outlooks}
As this paper demonstrated regular CNN architectures have their limitations.
Compared to referenced papers and performance measures of SotA architectures give reason to believe that the resulting model is not yet the perfect candidate to solve the here described task.
Due to time constraints of this thesis several approaches have not been tested or analyzed further.
Albeit, the here assembled architectures do not reach good performances, the outlook that is given by the results of a naively trained YOLOv5 is outstanding.
Compared to the other Backbone-Head architectures, the multi object detection by YOLO is a task none of the other ones can currently perform.
Furthermore, it is impossible to ever achieve multi object detection, with the current architecture of these models, because each of them is predicting one and only one BB per image.
Hence, the next steps should be to perform HPO for YOLOv5.
Another open question that seeks an answer is the slow inference time of YOLO, this should be target of an investigation, as well.

\section{Acknowledgement}
This thesis was supported by my great colleagues at \textit{KInsecta}, that welcomed me with open arms from the beginning and allowed me to take over the here described research.
I would like to express my appreciation to Prof. Dr. Frank Haußer and M. Sc. Teodor Chiaburu for the helpful discussions and support during this research.
Additionally, I would like to deeply thank my fiance Roberta, that has been under a lot of pressure and was forced to undergo my mood swings, over the last three months.
